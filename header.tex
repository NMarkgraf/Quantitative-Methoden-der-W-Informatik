% ===========================================================================
% header.tex (Release 5.0.3)
% ==========-----------------------------------------------------------------
%
% (C) in 2017-22 by Norman Markgraf (nmarkgraf(at)hotmail(dot)com)
%
% 17. Jan. 2019 (nm) Some tiny changes.
% 05. Jun. 2020 (nm) Die "columns"-Umgebung hacken, so das onlytextwidth
%                    zum default wird. Damit ragen die Spalten nicht mehr
%                    links und rechts aus dem Frame heraus.
% 31. Aug. 2020 (nm) Inhaltverzeichnis jetzt als Buttons (2.3)
% 23. Aug. 2021 (nm) Buttons nun als Boxen mit Umbruch (2.3.1)
% 21. Nov. 2021 (nm) Anpassungen für das DLS Layout
% 09. Feb. 2022 (nm) Weitere Anpassungen an das neue DLS Layout (5.0.2)
% 10. Feb. 2022 (nm) Von typeout und wlog auf message umgestellt. (5.0.3)
%
% ---------------------------------------------------------------------------

\message{Starting 'header.tex' ----------------------------------------------}

%\def\logowidth{1.3cm}
%\def\logoheight{1.3cm}
%\def\logoxshift{-0.85cm}
%\def\logoyshift{-0.85cm}


% Disable ligatures, helpful for some converting issues (ligature glyphs can raise problems):
%\usepackage{microtype}
%\DisableLigatures{encoding = *, family = *}


% Provide Option to change "€" in "EUR", as the "€" glyph can be problematic in some converting formats:
%\newcommand{\oureuro}{\xspace{}EUR\xspace{}}  %für SMALLPDF-Konvertierung
\newcommand{\oureuro}{\text{€}}   %für normales TeX-PDF
\let\euro\oureuro
%\newcommand{\oureuro}{\EUR{}}   %für normales TeX-PDF

\newcommand{\handleSection}{%
  %
  % New section, not section in headline
  %
  \setbeamertemplate{headline}[nosectioninhead]
  %
  \begin{frame}
    \frametitle{\phantom{ÜgyT}}
    \vspace*{1.2em}
    \vspace*{7cm}
    \begin{tikzpicture}[remember picture, overlay]
        \node[% section title
          %xshift=\logoxshift-6.6cm,
          xshift=\logoxshift-0.5\linewidth,
          yshift=0.4\paperheight, %3.8cm,
          inner sep=0pt
        ] at (current page.south east){%
            \makebox[\sectiontitleboxwidth][r]{%
              {%
                \usebeamercolor{sectionnumber title}%
                \usebeamerfont{sectionnumber title}\insertsectionnumber\hspace*{0.15em}%
              }%
              {%
                \usebeamercolor[fg]{section title}%
                \usebeamerfont{section title}\NoHyper\insertsectionhead\endNoHyper%
              }
            }% makebox
        };%
    \end{tikzpicture}%
  \end{frame}
  % Reset headline. Put section in head
  \setbeamertemplate{headline}[sectioninhead]
}

\newcommand{\handleSectionStar}{%
  %
  % New section, not section in headline
  %
  \setbeamertemplate{headline}[nosectioninhead]
  %
  \begin{frame}
    \frametitle{\phantom{ÜgyT}}
    \vspace*{1.2em}
    \vspace*{7cm}
    \begin{tikzpicture}[remember picture, overlay]
        \node[% section title
          %xshift=\logoxshift-6.6cm,
          xshift=\logoxshift-0.5\linewidth,
          yshift=0.4\paperheight, %3.8cm,
          inner sep=0pt
        ] at (current page.south east){%
            \makebox[\sectiontitleboxwidth][r]{%
              {%
                \usebeamercolor[fg]{section title}%
                \usebeamerfont{section title}\NoHyper\insertsectionhead\endNoHyper%
              }
            }% makebox
        };
    \end{tikzpicture}%
  \end{frame}
  % Reset headline. Put section in head
  \setbeamertemplate{headline}[sectioninhead]

}

\newcommand{\handleSectionDLS}{%
  %
  % New section, not section in headline
  %
  \setbeamertemplate{footline}[nofootline]
  %
  \begin{frame}
    \begin{tikzpicture}[remember picture, overlay]
        \node[inner sep=0pt] at (current page.center){%
            \includegraphics[width=\myPaperWidth, height=\myPaperHeight]{images/NPBT-FOM-DLS-section-background.png}%
          };%--
        \node[% section : section number
          xshift=3.85cm, %xshift=0.19\linewidth,
          yshift=0.32\paperheight, %3.8cm,
          inner sep=0pt
        ] at (current page.south west){%
            %%\fboxrule1.0mm%
            %%\fbox{%
            \makebox(2.92cm, 1.75cm)[c]{% 3cm = 8,5cm => x = 4,2
                \usebeamercolor[fg]{sectionnumber title}%
                \usebeamerfont{sectionnumber title}\insertsectionnumber%
            }% Ende makebox
            %%}% Ende fbox!
        };%
        \node[% section : section title
          xshift=14.3cm,%xshift=-0.415\linewidth,
          yshift=0.32\paperheight, %3.8cm,
          inner sep=0pt
        ] at (current page.south west){%
            %%\fboxrule1.0mm%
            %%\fbox{%
            %\makebox[\sectiontitleboxwidth][l]{%
            \makebox(16.2cm, 1.75cm)[l]{%
              {%
                \usebeamercolor[fg]{section title}%
                \usebeamerfont{section title}\NoHyper\insertsectionhead\endNoHyper%
              }
            }% Ende makebox
            %%}% Ende fbox!
        };%
    \end{tikzpicture}%
  \end{frame}
  % Reset headline. Put section in head
  \setbeamertemplate{footline}[normalDLS]
}

\newcommand{\handleSectionStarDLS}{%
  %
  % New section, not section in headline
  %
  \setbeamertemplate{footline}[nofootline]
  %
  \begin{frame}
    \begin{tikzpicture}[remember picture, overlay]
        \node[inner sep=0pt] at (current page.center){%
            \includegraphics[width=\myPaperWidth, height=\myPaperHeight]{images/NPBT-FOM-DLS-section-nonumber-background.png}%
          };%--
        \node[% section : section title
          xshift=14.3cm,%xshift=-0.415\linewidth,
          yshift=0.32\paperheight, %3.8cm,
          inner sep=0pt
        ] at (current page.south west){%
            %%\fboxrule1.0mm%
            %%\fbox{%
            %\makebox[\sectiontitleboxwidth][l]{%
            \makebox(16cm, 1.75cm)[l]{%
              {%
                \usebeamercolor[fg]{section title}%
                \usebeamerfont{section title}\NoHyper\insertsectionhead\endNoHyper%
              }
            }% Ende makebox
            %%}% Ende fbox!
        };%
    \end{tikzpicture}%
  \end{frame}
  % Reset headline. Put section in head
  \setbeamertemplate{footline}[normalDLS]
}


\AtBeginSection[\ifDLS{\handleSectionStarDLS}\else{\handleSectionStar}\fi]{\ifDLS{\handleSectionDLS}\else{\handleSection}\fi}

\newcommand{\handleSubsection}{%
  %
  % New subsection, not section in headline
  %
  \setbeamertemplate{headline}[nosectioninhead]
  %
  \begin{frame}
    \frametitle{\phantom{ÜgyT}}
    \vspace*{1.2em}
    \vspace*{7cm}
    \begin{tikzpicture}[remember picture, overlay]
        \node[% section title
          xshift=\logoxshift-0.5\linewidth,
          yshift=0.4\paperheight, %3.8cm,
          %anchor=south east,
          %yshift=3.8cm,
          inner sep=0pt
        ] at (current page.south east){%
%          \fbox{
            \makebox[\sectiontitleboxwidth][r]{%
              {%
                \usebeamercolor[fg]{sectionnumber title}%
                \usebeamerfont{sectionnumber title}\insertsectionnumber.%\insertsubsectionnumber\hspace*{0.15em}%
              }%
              {%
                \usebeamercolor[fg]{subsectionnumber title}%
                \usebeamerfont{subsectionnumber title}\insertsubsectionnumber\hspace*{0.15em}%
              }%
              {%
                \usebeamercolor[fg]{subsection title}%
                \usebeamerfont{subsection title}\NoHyper\insertsubsectionhead\endNoHyper%
              }
            }% makebox
%          }% fbox
        };
    \end{tikzpicture}%
  \end{frame}
  % Reset headline. Put section in head
  \setbeamertemplate{headline}[sectioninhead]
%  \addtocounter{framenumber}{-1}% If you don't want them to affect the slide number

}

\newcommand{\handleSubsectionDLS}{%
  %
  % New subsection, not section in headline
  %
  \setbeamertemplate{footline}[nofootline]
  %
   \begin{frame}
    \begin{tikzpicture}[remember picture, overlay]
        \node[inner sep=0pt] at (current page.center){%
            \includegraphics[width=\myPaperWidth, height=\myPaperHeight]{images/NPBT-FOM-DLS-section-background.png}%
          };%--
        \node[% section : section number
          xshift=3.85cm, %xshift=0.19\linewidth,
          yshift=0.32\paperheight, %3.8cm,
          inner sep=0pt
        ] at (current page.south west){%
            %%\fboxrule1.0mm%
            %%\fbox{%
            \makebox(2.92cm, 1.75cm)[c]{% 3cm = 8,5cm => x = 4,2
                    {%
                        \usebeamercolor[fg]{sectionnumber title}%
                        \usebeamerfont{sectionnumber title}\insertsectionnumber.%
                    }%
                    {%
                        \usebeamercolor[fg]{subsectionnumber title}%
                        \usebeamerfont{subsectionnumber title}\insertsubsectionnumber\hspace*{0.15em}%
                    }%
                }%
        };%
        \node[% section : section title
          xshift=14.3cm,%xshift=-0.415\linewidth,
          yshift=0.32\paperheight, %3.8cm,
          inner sep=0pt
        ] at (current page.south west){%
            %%\fboxrule1.0mm%
            %%\fbox{%
            %\makebox[\sectiontitleboxwidth][l]{%
            \makebox(16cm, 1.75cm)[l]{%
              {%
                \usebeamercolor[fg]{section title}%
                \usebeamerfont{section title}\NoHyper\insertsectionhead\endNoHyper%
              }%
            }% makebox
            %%}% Ende fbox
        };%
    \end{tikzpicture}%
  \end{frame}
  % Reset headline.
  \setbeamertemplate{footline}[normalDLS]
%  \addtocounter{framenumber}{-1}% If you don't want them to affect the slide number
}

\newcommand{\handleSubsectionStar}{%
  %
  % New subsection, not section in headline
  %
  \setbeamertemplate{headline}[nosectioninhead]
  %
  \begin{frame}
    \frametitle{\phantom{ÜgyT}}
    \vspace*{1.2em}
    \vspace*{7cm}
    \begin{tikzpicture}[remember picture, overlay]
      \node[% section title
          xshift=\logoxshift-0.5\linewidth,
          yshift=0.4\paperheight, %3.8cm,
          %anchor=south east,
          %yshift=3.8cm,
          inner sep=0pt
      ] at (current page.south east){%
%          \fbox{
            \makebox[\sectiontitleboxwidth][r]{%
              {%
                \usebeamercolor[fg]{subsection title}%
                \usebeamerfont{subsection title}\NoHyper\insertsubsectionhead\endNoHyper%
              }
            }% makebox
%          }% fbox
    };% war mal -3pt
    \end{tikzpicture}%
  \end{frame}
  % Reset headline. Put section in head
  \setbeamertemplate{headline}[sectioninhead]
}

\newcommand{\handleSubsectionStarDLS}{%
  %
  % No footline!
  %
  \setbeamertemplate{footline}[nofootline]
  %
   \begin{frame}
    \begin{tikzpicture}[remember picture, overlay]
        \node[inner sep=0pt] at (current page.center){%
            \includegraphics[width=\myPaperWidth, height=\myPaperHeight]{images/NPBT-FOM-DLS-section-background.png}%
          };%--
        \node[% section : section title
          xshift=14.3cm,%xshift=-0.415\linewidth,
          yshift=0.32\paperheight, %3.8cm,
          inner sep=0pt
        ] at (current page.south west){%
            %%\fboxrule1.0mm%
            %%\fbox{%
            %\makebox[\sectiontitleboxwidth][l]{%
            \makebox(16cm, 1.75cm)[l]{%
              {%
                \usebeamercolor[fg]{section title}%
                \usebeamerfont{section title}\NoHyper\insertsectionhead\endNoHyper%
              }%
            }% makebox
            %%}% Ende fbox
        };%
    \end{tikzpicture}%
  \end{frame}
  % Reset footline!
  \setbeamertemplate{footline}[normalDLS]
%  \addtocounter{framenumber}{-1}% If you don't want them to affect the slide number
}


\AtBeginSubsection[\ifDLS{\handleSubsectionStarDLS}\else{\handleSubsectionStar}\fi]{\ifDLS{\handleSubsectionDLS}\else{\handleSubsection}\fi}

%
% Remove unwanted toc entries:
%
\renewcommand\addcontentsline[3]{%
\ifthenelse{\equal{#1}{toc}}{\relax}{\addtocontents{#1}{\protect\contentsline{#2}{#3}}}
}


\newcommand{\setnormaltoc}{%
    \message{+++++ Switch to Old Table of Content!}%
%
% Altes Inhaltsverzeichnis
%
\setbeamertemplate{section in toc}{%
    \leavevmode\leftskip=2.75ex%
  \llap{%
    \usebeamerfont*{section number projected}%
    \usebeamercolor[bg]{section number projected}%
    \vrule width2.50ex height2.05ex depth.4ex%
    \hskip-2.50ex%
    \hbox to2.50ex{\hfil\color{fg}\inserttocsectionnumber\hfil}}%
  \kern1.35ex\inserttocsection\par}
}

\newcommand{\setmoderntoc}{%
    \message{+++++ Switch to New Table of Content!}%
    %
    % Neues Inhaltsverzeichnis
    %
    \setbeamertemplate{button}{\tikz
      \node[
          inner xsep=10pt,    %
          outer sep=0pt,      %
%          draw=FOMTOCBackgroundColor!80,  %
%          fill=FOMTOCBackgroundColor!50,  %
          draw=FOMTOCBackgroundColor!99,  %
          fill=FOMTOCBackgroundColor!90,  %
          minimum size=5.7em, %
          %text width=5.9cm,   %
          text width=6.5cm,   %
          rounded corners=4pt %
       ] {\large\insertbuttontext};%
    } % End setbeamertemplate
    
    \setbeamertemplate{section in toc}{%
        \beamerbutton{%
            %\fbox{%
            \usebeamercolor[fg]{FOMTOCSectionNumberColor}%
            \inserttocsectionnumber.~ %
            %}
            %\vspace*{-.75pt}%
            %\vskip-0.25cm%
            %\fbox{%
            \parbox{%
%                \dimexpr\linewidth-4\fboxsep-4\fboxrule%
                \dimexpr\linewidth-5\fboxsep-5\fboxrule-3pt%
             }{%
                \usebeamercolor[fg]{FOMTOCSectionTitleColor}%
                \raggedright\nohyphens{\inserttocsection}%
            }% 
            %}% End fbox
        }% End beamerbutton
    }% End setbeamertemplate
}% End newcommand


%\makeatletter
\LetLtxMacro{\oldtableofcontents}{\tableofcontents}
\renewcommand{\tableofcontents}[1][]{%
  \message{New Table of Content in use!}%
  \setlength{\columnsep}{0.2cm}%
  \raggedcolumns
  \setkeys{beamerframe}{noframenumbering=true}
  \begin{multicols}{2}
        \vspace*{-4pt}%
        \hypersetup{linkbordercolor=white, linkcolor=white, colorlinks=false, hidelinks}% hidelinks
        \oldtableofcontents[#1]
  \end{multicols}%
  \message{New Table of Content in use! **ENDE**}%
}
%\makeatother

%\setnormaltoc

%
% BUGFIX für caption bug in 2018!
%
% see: https://tex.stackexchange.com/questions/426088/texlive-pretest-2018-beamer-and-subfig-collide/426090#426090
% see: https://www.mrunix.de/forums/showthread.php?77329-Undefined-control-sequence-magyar-captionfix-beim-Laden-von-subfig-in-beamer
% see: https://gitlab.com/axelsommerfeldt/caption/issues?state=opened
%
\makeatletter
\let\@@magyar@captionfix\relax
\makeatother



%
% Default für columns auf "onlytextwidth" setzen, damit die Spalten besser
% ins Bild passen.
% Quelle: https://tex.stackexchange.com/questions/204022/make-onlytextwidth-default-width-of-columns
%
\makeatletter
\long\def\beamer@newenvnoopt#1#2#3#4{%
    \expandafter\renewcommand\expandafter<\expandafter>\csname#1\endcsname[#2]{#3}%<- here
    \expandafter\long\expandafter\def\csname end#1\endcsname{#4}%
}
\long\def\beamer@newenvopt#1#2[#3]#4#5{%
    \expandafter\renewcommand\expandafter<\expandafter>\csname#1\endcsname[#2][#3]{#4}%<- here
    \expandafter\long\expandafter\def\csname end#1\endcsname{#5}%
}

\renewenvironment<>{columns}[1][]{%
    \begin{actionenv}#2%
        \def\beamer@colentrycode{%
            \hbox to\textwidth\bgroup%
            \leavevmode%
            \hskip-\beamer@leftmargin%
            \nobreak%
            \beamer@tempdim=\textwidth%
            \advance\beamer@tempdim by\beamer@leftmargin%
            \advance\beamer@tempdim by\beamer@rightmargin%
            \hbox to\beamer@tempdim\bgroup%
            \hbox{}\hfill\ignorespaces}%
        \def\beamer@colexitcode{\egroup%
            \nobreak%
            \hskip-\beamer@rightmargin\egroup}%
        \ifbeamer@centered\setkeys{beamer@col}{c}\else\setkeys{beamer@col}{t}\fi%
        \setkeys{beamer@col}{#1, onlytextwidth}% added "onlytextwidth"
        \par%
        \beamer@colentrycode%
        \def\beamer@colclose{}\ignorespaces}%
    {\beamer@colclose\def\beamer@colclose{}\beamer@colexitcode\end{actionenv}}%
\makeatother

%%%%%%%% TEST %%%%%%%%
\makeatletter
\let\oldsectionintoc\beamer@sectionintoc%
\def\beamer@sectionintoc#1#2#3#4#5{%
%    \nonstopmode% Ich schäme mich für diese Zeile, aber ich finde den Bug nicht!
%    % 7 / 13 / 19 
%    % 6 / 11 / 16
    \ifnum#1=6 \vfill\columnbreak \else \relax \fi%
%%    \ifnum#1=11 \vfill\columnbreak\expandafter\end\expandafter{multicols}%
%%    \expandafter\framebreak\vspace*{2pt}\expandafter\begin\expandafter{multicols}{2}%
    \ifnum#1=11 \vfill\egroup\endmulticols%
    \framebreak\vspace*{2pt}\multicols{2}[]\bgroup%
    \makeatletter%
    \def\beamer@toc@cs{show}%
    \def\beamer@toc@os{show}%
    \def\beamer@tocsections{<*>}%
    \setkeys{beamertoc}{hideallsubsections}%
    \hypersetup{linkbordercolor=white, linkcolor=white, hidelinks}%
    \else \relax%
    \fi%
    \ifnum#1=16 \vfill\columnbreak \else \relax \fi%
    \oldsectionintoc {#1}{#2}{#3}{#4}{#5}%
%    \batchmode% Ich schäme mich für diese Zeile, aber ich finde den Bug nicht!
}%
\makeatother


\InputIfFileExists{include-notes.tex}{\relax}{\relax}
\InputIfFileExists{header_customs.tex}{\relax}{\relax}

\message{Ending 'header.tex' -----------------------------------------------}
% ===========================================================================
