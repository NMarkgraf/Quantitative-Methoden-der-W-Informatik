% --------------------------------------------------------------------
% person.tex
% =============-------------------------------------------------------
% (C) in 2015 by Norman markgraf
%  
% Release 1.0 (nm) 16.06.2015 Initial Concept
% Release 1.1 (nm) 21.06.2015 First useable release
% Release 1.2 (nm) 23.06.2015 New approach. Just a template now.
% Release 1.3 (nm) 29.07.2015 H. Rutishauser (LR/LU-Algorithem) added
% Release 1.4 (nm) 11.01.2016 Laplace added
% Release 1.5 (nm) 12.01.2016 Some changes to make it more usable
% --------------------------------------------------------------------
% Makro für Personen
% --------------------------------------------------------------------

% --------------------------------------------------------------------
% Path to the images of the persons
%
\newcommand*{\pathtopersonimages}{images/portrait/}

% --------------------------------------------------------------------
% Width of the whole person box
%
\newlength{\personboxwidth}
\setlength{\personboxwidth}{2.4cm}

% --------------------------------------------------------------------
% Width of person image inside of the Person Box
%
\newlength{\personimangewidth}
\setlength{\personimangewidth}{2.2cm}

% --------------------------------------------------------------------
%
%
\newenvironment{personblock}{%
  \setbeamercolor{block body}{bg=blue!25,fg=black!85}
  \begin{block}}{\end{block}}

% --------------------------------------------------------------------
% Mark before a birth date
\newcommand*{\birthmark}{*}

% --------------------------------------------------------------------
% Mark before a death date
\newcommand*{\deathmark}{\dag}

% --------------------------------------------------------------------
% Seperation between birth- or deathmark and date.
% The default \thinspace looks better than \enskip, i think.
\newcommand*{\markseperation}{\thinspace}

% --------------------------------------------------------------------
% Path to the images of the persons
\newcommand{\person}[5]{%
% --------------------------------------------------------------------
%
% 1 : file name of the image
% 2 : full name
% 3 : year of birth and place
% 4 : year of death and place
% 5 : link to (german) wikipedia
%
% todo:
%
% 6 : Licence 
% 
% 'figure' list.
%
%	\colorbox{blue!10}{%
	\vspace*{-1em}
	\begin{center}
		\begin{minipage}{\personboxwidth}
			\begin{personblock}\TINY
			\begin{center}
				\href{#5}{\includegraphics[width=\personimangewidth]{\pathtopersonimages#1}}
%			\texttt{\detokenize{#1}} 
			\end{center}\TINY%
%			\vspace*{-1.8em}
			\Tiny		
			
%			{\textsc{#2}}
			{#2}
			
			{\birthmark\markseperation#3}
		
			{\deathmark\markseperation#4}
			\end{personblock}
		\end{minipage}
%	}%
	\end{center}
}%
%
% --------------------------------------------------------------------
% Begin of the "database"
\newcommand{\personDB}[1]{%
%----------------------------------------------------------------------------------------
	\ifthenelse{\equal{#1}{Kolmogorow}}{% Andrei Nikolajewitsch Kolmogorow (russisch Андре́й Никола́евич Колмого́ров  Aussprache?/i, wissenschaftliche Transliteration Andrej Nikolaevič Kolmogorov; * 12.jul./ 25. April 1903greg. in Tambow; † 20. Oktober 1987 in Moskau)
		\person{220px-Andrej_Nikolajewitsch_Kolmogorov}{Andrei N. Kolmogorow}{1903 in Tambow}{1987 in Moskau}{%
		https://de.wikipedia.org/wiki/Andrei_Nikolajewitsch_Kolmogorow}
	}{\relax}
%----------------------------------------------------------------------------------------
	\ifthenelse{\equal{#1}{Venn}}{% John Venn Junior (* 4. August 1834 in Kingston upon Hull; † 4. April 1923 in Cambridge) 
		\person{John_Venn}{John Venn}{1834 in Kingston upon Hull}{1923 in Cambridge}{%
		https://de.wikipedia.org/wiki/John_Venn
		}
	}{\relax}
%----------------------------------------------------------------------------------------
	\ifthenelse{\equal{#1}{Newton}}{% Sir Isaac Newton [ˌaɪzək ˈnjuːtən] (* 25. Dezember 1642jul./ 4. Januar 1643greg. in Woolsthorpe-by-Colsterworth in Lincolnshire; † 20. März 1726jul./ 31. März 1727greg. in Kensington)
		\person{180px-Isaac_Newton}{Sir Isaac Newton}{1642 in Lincolnshire}{1726 in Kensington}{%
		https://de.wikipedia.org/wiki/Isaac_Newton}
	}{\relax}
%----------------------------------------------------------------------------------------
	\ifthenelse{\equal{#1}{Peano}}{% Giuseppe Peano (* 27. August 1858 in Spinetta, heute Teil von Cuneo, Piemont; † 20. April 1932 in Turin)
		\person{Giuseppe_Peano}{Giuseppe Peano}{1858 in Spinetta}{1932 in Turin}{%
		https://de.wikipedia.org/wiki/Giuseppe_Peano}
	% „Giuseppe Peano“ von Unbekannt - School of Mathematics and Statistics, University of St Andrews, Scotland [1]. Lizenziert unter Gemeinfrei über Wikimedia Commons - https://commons.wikimedia.org/wiki/File:Giuseppe_Peano.jpg#/media/File:Giuseppe_Peano.jpg
	}{\relax}
%----------------------------------------------------------------------------------------
	\ifthenelse{\equal{#1}{Vieta}}{% François Viète oder Franciscus Vieta, wie er sich in latinisierter Form nannte (* 1540 in Fontenay-le-Comte; † 13. Dezember, nach anderen Quellen 23. Februar 1603 in Paris)
		\person{Francois_Viete}{Fran\c{c}ois Vi\`{e}te (aka Franciscus Vieta)}{1540 in Fontenay-le-Comte}{1603 in Paris}{%
		 https://de.wikipedia.org/w/index.php?title=Fran\%C3\%A7ois_Vi\%C3\%A8te}
	}{\relax}
%----------------------------------------------------------------------------------------
	\ifthenelse{\equal{#1}{Fibunacci}}{% Leonardo da Pisa, auch Fibonacci genannt (* um 1170 in Pisa; † nach 1240 ebenda)
		\person{Fibonacci}{Leonardo da Pisa, auch Fibonacci genannt}{1170 in Pisa}{1240 Pisa}{%
		https://de.wikipedia.org/wiki/Leonardo_Fibonacci}
	}{\relax}
%----------------------------------------------------------------------------------------
	\ifthenelse{\equal{#1}{vonNeumann}}{% John von Neumann, ungarisch margittai Neumann János (* 28. Dezember 1903 in Budapest (Österreich-Ungarn) als János Lajos Neumann; † 8. Februar 1957 in Washington, D.C.)
		\person{200px-JohnvonNeumann-LosAlamos}{John von Neumann}{1903 in Budapest}{1957 in Washington, D.C.}{%
		https://de.wikipedia.org/wiki/John_von_Neumann}
	}{\relax}	
%----------------------------------------------------------------------------------------
	\ifthenelse{\equal{#1}{Sylvester}}{% James Joseph Sylvester (* 3. September 1814 in London; † 15. März 1897 ebenda) war ein britischer Mathematiker.
		\person{James_Joseph_Sylvester}{James Joseph Sylvester}{1814 in London}{1897 in London}{%
		https://de.wikipedia.org/wiki/James_Joseph_Sylvester}
		%„James Joseph Sylvester“ von from:http://en.wikipedia.org/wiki/Image:Untitled04.jpg. Lizenziert unter Gemeinfrei über Wikimedia Commons - https://commons.wikimedia.org/wiki/File:James_Joseph_Sylvester.jpg#/media/File:James_Joseph_Sylvester.jpg
	}{\relax}
%----------------------------------------------------------------------------------------
	\ifthenelse{\equal{#1}{Poincare}}{% Jules Henri Poincaré [pwɛ̃kaˈʀe] (* 29. April 1854 in Nancy; † 17. Juli 1912 in Paris) 		
		\person{Young_Poincare}{Jules Henri Poincar\'{e}}{1854 in Nancy}{1912 in Paris}{%
		https://de.wikipedia.org/wiki/Henri_Poincaré}
	% „Young Poincare“ von Eugène Pirou (1841–1909) [1]. Lizenziert unter Gemeinfrei über Wikimedia Commons - https://commons.wikimedia.org/wiki/File:Young_Poincare.jpg#/media/File:Young_Poincare.jpg
	}{\relax}
%----------------------------------------------------------------------------------------
	\ifthenelse{\equal{#1}{Cantor}}{% Georg Ferdinand Ludwig Philipp Cantor [ɡ̥eˈɔʁk (ˈfɛʁdinant ˈluːtvɪç ˈfiːlɪp) ˈkʰantɔʁ] (* 19. Februarjul./ 3. März 1845greg. in Sankt Petersburg; † 6. Januar 1918 in Halle an der Saale) 
		\person{180px-Georg_Cantor}{Georg Cantor}{1845 in Sankt Petersburg}{1918 in Halle an der Saale}{%
		https://de.wikipedia.org/wiki/Georg_Cantor}
	}{\relax}
%----------------------------------------------------------------------------------------
	\ifthenelse{\equal{#1}{Gauss}}{% Johann Carl Friedrich Gauß (latinisiert Carolus Fridericus Gauss; * 30. April 1777 in Braunschweig; † 23. Februar 1855 in Göttingen) war ein deutscher Mathematiker, Astronom, Geodät und Physiker.
		\person{Carl_Friedrich_Gauss}{Johann Carl Friedrich Gau{\ss}}{1777 in Braunschweig}{1855 in Göttingen}{%
		https://de.wikipedia.org/wiki/Carl_Friedrich_Gauß}
		% „Carl Friedrich Gauss“ von Gottlieb BiermannA. Wittmann (photo) - Gauß-Gesellschaft Göttingen e.V. (Foto: A. Wittmann).. Lizenziert unter Gemeinfrei über Wikimedia Commons - https://commons.wikimedia.org/wiki/File:Carl_Friedrich_Gauss.jpg#/media/File:Carl_Friedrich_Gauss.jpg
	}{\relax}
%----------------------------------------------------------------------------------------
	\ifthenelse{\equal{#1}{Pascal}}{% 
		\person{180px-Blaise_Pascal.jpg}{Blaise Pascal}{1623 in Clermont-Ferrandx}{1662 in Paris}{%
		https://de.wikipedia.org/wiki/Blaise_Pascal}
	}{\relax}
%----------------------------------------------------------------------------------------
	\ifthenelse{\equal{#1}{Goodstein}}{% Reuben Louis Goodstein (* 15. Dezember 1912 in London; † 8. März 1985 in Leicester)
		\person{260px-Goodstein}{Reuben Louis Goodstein}{1912 in London}{1985 in Leicester}{%
		https://de.wikipedia.org/wiki/Goodstein-Folge}
	}{\relax}
%----------------------------------------------------------------------------------------
	\ifthenelse{\equal{#1}{Bayes}}{% Thomas Bayes [beɪz] (* um 1701 in London; † 7. April 1761[1] in Tunbridge Wells)
		\person{Thomas_Bayes}{Thomas Bayes}{um 1701 in London}{1761 in Tunbridge Wells}{%
		https://de.wikipedia.org/wiki/Thomas_Bayes}
		% „Thomas Bayes“ von unknown - [2][3]. Lizenziert unter Gemeinfrei über Wikimedia Commons - https://commons.wikimedia.org/wiki/File:Thomas_Bayes.gif#/media/File:Thomas_Bayes.gif
	}{\relax}
%----------------------------------------------------------------------------------------
	\ifthenelse{\equal{#1}{Leibniz}}{% Gottfried Wilhelm Leibniz (* 21. Junijul./ 1. Juli 1646greg. in Leipzig; † 14. November 1716 in Hannover) 
		\person{Gottfried_Wilhelm_von_Leibniz}{Gottfried Wilhelm Leibniz}{1646 in Leipzig}{1716 in Hannover}{%
		https://de.wikipedia.org/wiki/Gottfried_Wilhelm_Leibniz}
		% „Gottfried Wilhelm von Leibniz“ von Christoph Bernhard Francke - /gbrown/philosophers/leibniz/BritannicaPages/Leibniz/LeibnizGif.html. Lizenziert unter Gemeinfrei über Wikimedia Commons - https://commons.wikimedia.org/wiki/File:Gottfried_Wilhelm_von_Leibniz.jpg#/media/File:Gottfried_Wilhelm_von_Leibniz.jpg
	}{\relax}
%----------------------------------------------------------------------------------------
	\ifthenelse{\equal{#1}{Nickel}}{% Karl Leberecht Emil Nickel (* 9. Februar 1924 in Tübingen; † 1. Januar 2009 in Freiburg im Breisgau) 
		\person{Karl_Nickel}{Karl Leberecht Emil Nickel}{1924 in Tübingen}{2009 in Freiburg im Breisgau}{%
		https://de.wikipedia.org/wiki/Karl_Nickel}
		% „Karl Nickel“ von Konrad Jacobs - http://owpdb.mfo.de/detail?photo_id=3048. Lizenziert unter CC BY-SA 2.0 de über Wikimedia Commons - https://commons.wikimedia.org/wiki/File:Karl_nickel.jpg#/media/File:Karl_nickel.jpg
	}{\relax}
%----------------------------------------------------------------------------------------
	\ifthenelse{\equal{#1}{Rutishauser}}{% Heinz Rutishauser (* 30. Januar 1918 in Weinfelden, Schweiz; † 10. November 1970 in Zürich)  
		\person{1960_rutishauser}{Heinz Rutishauser}{1918 in Weinfelden, Schweiz}{1970 in Zürich}{%
		https://de.wikipedia.org/wiki/Heinz_Rutishauser}
		% 
	}{\relax}
%----------------------------------------------------------------------------------------
	\ifthenelse{\equal{#1}{Dedekind}}{% Heinz Rutishauser (* 30. Januar 1918 in Weinfelden, Schweiz; † 10. November 1970 in Zürich)  
		\person{Dedekind-2}{Richard Dedekind}{1831 in Braunschweig}{1916 in Braunschweig}{%
		https://de.wikipedia.org/wiki/Richard_Dedekind}
		% Julius Wilhelm Richard Dedekind (* 6. Oktober 1831 in Braunschweig; † 12. Februar 1916 ebenda) 
		% Alternative Bilder: Dedkind, Dedekind-2, Richard_Dedekind_1900s
	}{\relax}
%----------------------------------------------------------------------------------------
	\ifthenelse{\equal{#1}{Horner}}{% William George Horner (* 1786 in Bristol; † 22. September 1837 in Bath) war ein englischer Mathematiker. 
		\person{Horner}{William George Horner}{1786 in Bristol}{1837 in Bath}{%
		https://de.wikipedia.org/wiki/William_George_Horner}
	}{\relax}
%----------------------------------------------------------------------------------------
	\ifthenelse{\equal{#1}{Maurolicus}}{% Franciscus Maurolicus (* 16. September 1494 in Messina; † 21./ 22. Juli 1575 bei Messina; auch Francesco Maurolico, griech. Frangiskos Maurolykos) war ein bedeutender Universalgelehrter des 16. Jahrhunderts.  
		\person{Maurolico}{Franciscus Maurolicus}{1494 in Messina}{1575 bei Messina}{%
		https://de.wikipedia.org/wiki/Franciscus_Maurolicus}
		% „Maurolico“. Lizenziert unter Gemeinfrei über Wikimedia Commons - https://commons.wikimedia.org/wiki/File:Maurolico.jpg#/media/File:Maurolico.jpg
		%\person{MAUROLICO_FRANCESCO2}{Franciscus Maurolicus}{1494 in Messina}{1575 bei Messina}
		%"MAUROLICO FRANCESCO2" by Unknown - http://badigit.comune.bologna.it/facies/dettagli_ritrattati.aspx?IDImg=7748. Licensed under Public Domain via Commons - https://commons.wikimedia.org/wiki/File:MAUROLICO_FRANCESCO2.JPG#/media/File:MAUROLICO_FRANCESCO2.JPG
	}{\relax}
%----------------------------------------------------------------------------------------
	\ifthenelse{\equal{#1}{Euler}}{% Leonhard Euler (lat. Leonhardus Eulerus; * 15. April 1707 in Basel; † 7. Septemberjul./ 18. September 1783greg. in Sankt Petersburg)
		\person{LeonhardEulerByHandmann}{Leonhard Euler}{1707 in Basel}{1783 in Sankt Petersburg}{%
		https://de.wikipedia.org/wiki/Leonhard_Euler}
		% {Von Jakob Emanuel Handmann - Cropped from: http://www.euler-2007.ch/doc/Bild0015.pdf, Gemeinfrei }
	}{\relax}
%----------------------------------------------------------------------------------------
	\ifthenelse{\equal{#1}{Laplace}}{% Pierre-Simon (Marquis de) Laplace (* 28. März 1749[1] in Beaumont-en-Auge in der Normandie; † 5. März 1827 in Paris) war ein französischer Mathematiker, Physiker und Astronom. 
		\person{Pierre-Simon_Laplace}{Pierre-Simon (Marquis de) Laplace}{1749 in Beaumont-en-Auge}{1827 in Paris}{%
		https://de.wikipedia.org/wiki/Pierre-Simon_Laplace}
		% „Pierre-Simon Laplace“ von Sophie Feytaud (fl.1841) - This image appears identical to the cover image used by Gillispie et al. They cite the portrait as an 1842 posthumous portrait by Madame Feytaud, courtesy of the Académie des Sciences, Paris.. Lizenziert unter Gemeinfrei über Wikimedia Commons - https://commons.wikimedia.org/wiki/File:Pierre-Simon_Laplace.jpg#/media/File:Pierre-Simon_Laplace.jpg
	}{\relax}
%----------------------------------------------------------------------------------------
	\ifthenelse{\equal{#1}{Heron}}{% 
		\person{Hero_of_Alexandria}{Heron von Alexandria}{unbekannt}{unbekannt}{%
		https://de.wikipedia.org/wiki/Heron_von_Alexandria}
		% "Hero of Alexandria" by Unknown - http://www.xtec.es/~jcanadil/imatges/personatges/actius/Heron.jpg. Licensed under Public Domain via Commons - https://commons.wikimedia.org/wiki/File:Hero_of_Alexandria.png#/media/File:Hero_of_Alexandria.png
	}{\relax}
}
%
% End of the "database"
% --------------------------------------------------------------------